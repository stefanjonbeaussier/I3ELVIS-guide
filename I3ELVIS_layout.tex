% Chapter code layout

\section{Code Layout}

\subsection{Files layout}
 
Short description of the code files and their functions. \\
The code is divided into eight .c files 
   \begin{itemize}
    \item \textcolor{blue}{in3mg.c :} This is the file that will read the parameters files
    to write the initial condition file.  
    \item \textcolor{blue}{i3mg.c :} Main code file.
    \item  \textcolor{blue}{head3mg.c :} In this file are the global variable and function are declared (internal counters are declared inside the function or loop that use them). Most variables roles are describe in this file.
    \item \textcolor{blue}{load3mg.c :} In this file are the functions to read the input files and write their content to the code variables, and to save the results to the output files. The functions have a binary or text option but only the binary option should be used as the text option would slow the code significantly.
    \item \textcolor{blue}{move3mg.c :} Contain the function to resolve continuity-momentum equations. From the function to calculate the stress and strain to the multi-grid resolution functions.
    \item \textcolor{blue}{mark3mg.c :} Contain all the interpolation functions and the functions to advect the marker.
    \item \textcolor{blue}{gauss3mg.c :} others Gauss-Scheidel solvers
    \item \textcolor{blue}{heat3mg.c :} Solve heat conservation equation and using the thermodynamic database.      
    \end{itemize}
    To those eight files we can add three configuration files. \textcolor{blue}{mode.t3c} that contain all the configurations and parameters of the code that you will be running and the name of all the output files. \textcolor{blue}{init.t3c} contain the parameters defining the initial geometry of model (support rectangular, Ellipsoidal and cylindrical geometry), the different rock types and there rheological parameters.\textcolor{blue}{files.t3c} contain only the number of the next output file (or 0). \\  
Finally there is a large number of thermodynamics database that are used for proper calculation of temperature controlled by phase changes and rock behaviour in P,T,t conditions.

\subsection{Functions layout}

