% chapter: prn files

\section{Raw output files: .prn}
Valid for subduction/collision setup

\subsection{Part I: General Information}
\begin{table}[H]
\small
\centering
\begin{tabular}{p{3cm} p{6cm} p{3cm} l}
\toprule
Parameter & Description & Initial value & Unit\\
\midrule
xnumx & Number of nodes in x & $16N+5$ (4 multigrid levels) & - \\
ynumy & Number of nodes in y & $16N+5$ (4 multigrid levels) & - \\
znumz & Number of nodes in z & $16N+5$ (4 multigrid levels) & - \\
mnumx & Number markers per cell in x & - & - \\
mnumy & Number markers per cell in y & - & - \\
mnumz & Number markers per cell in z & - & - \\
xsize & Dimension of model in x & - & $[m]$ \\
ysize & Dimension of model in y & - & $[m]$ \\
zsize & Dimension of model in z & - & $[m]$ \\
pxinit & x-coordinate of initial pressure cell & $(xnumx-1)/2$ & - \\
pyinit & y-coordinate of initial pressure cell & $(ynumy-1)/2$ & - \\
pzinit & z-coordinate of initial pressure cell & - & - \\
pinit & Initial pressure in initial pressure cell & - & $[Pa]$ \\
GXKOEFF & Gravitational acceleration in x & - & $[m/s^2]$ \\
GYKOEFF & Gravitational acceleration in y & - & $[m/s^2]$ \\
GZKOEFF & Gravitational acceleration in z & - & $[m/s^2]$ \\
rocknum & \\
bondnum & \\
marknum & \\
n1 & \\
timesum & Starting time & - & $[years]$ \\
ival1 & \\
gridcur & \\
gridtot & \\
\bottomrule
\end{tabular}
\caption{prn-file general information block}
\label{tbl:prn_general_info}
\end{table}

\subsection{Part II: Rock type information
}
\begin{table}[H]
\small
\centering
\begin{tabular}{l p{8cm} l}
\toprule
Parameter & Description & Unit\\
\midrule
markn0 & Individual lower viscosity limit & $[Pa*s]$ \\ 
markn1 & Individual upper viscosity limit & $[Pa*s]$ \\ 
marks0 & Individual lower stress limit & $[Pa]$ \\ 
marks1 & Individual upper stress limit & $[Pa]$ \\ 
marknu & Newtonian viscosity & $[Pa^{MM}*s]$ \\ 
markdh & Activation energy & $[J]$\\ 
markdv & Activation volume & $[J/bar]$ \\ 
markss & Dislocation/diffusion creep stress threshold& $[Pa]$ \\ 
markmm & Stress exponent & $(Power)$ \\ 
markll & Pore fluid pressure factor & $(koef)$ \\ 
marka0 & Cohesion & $[Pa]$ \\
marka1 & Cohesion 2 & $[Pa]$ \\
markb0 & Sine of dry friction angle & $[-]$ \\
markb1 & Sine of dry friction angle 2 & $[-]$ \\
marke0 & Lower strain threshold for brittle/ductile transition & $[-]$ \\
marke1 & Upper strain threshold for brittle/ductile transition & $[-]$ \\ 
markro & Density & $[kg/M^3]$ \\ 
markbb & Density koef b & $[1/K]$ \\ 
markaa & Density koef a & $[1/kbar]$\\ 
markcp & Heat capacity  & $[J/kg]$ \\ 
markkt & Thermal conductivity & $[W/(m*K)]$ \\ 
markkf & Temperature dependency koef in conductivity & $[W/(m)]$ \\ 
markkp & Pressure dependency koef in conductivity & $[Pa^{-1}]$ \\ 
markht & heat generation & $[W/kg]$\\ 
\bottomrule
\end{tabular}
\caption{prn-file Rock type information block, see also tbl~\ref{tbl:rock_type_parameters}.}
\label{tbl:prn_rock_type_info}
\end{table}

\subsection{Part III: Nodes information}

For each node the following parameters are given. The order is the following, where $n$ is the number of nodes $nodenum$: $${pr_0,vx_0,...ht_0,pr_1,..,ht_1,..,pr_n,..,ht_n}$$

\begin{table}[H]
\small
\centering
\begin{tabular}{l l p{2cm}}
\toprule
Parameter & Description & Unit\\
\midrule
pr & Pressure & $[Pa]$ \\ 
vx & Velocity in x & $[m/s]$ \\ 
vy & Velocity in y & $[m/s]$ \\ 
vz & Velocity in z & $[m/s]$ \\ 
ro & Density & $[kg/m^3]$ \\ 
nu & Viscosity & $[Pa*s]$ \\ 
tk & Temperature & $[K]$ \\ 
cp & Heat capacity & $[J/kg]$ \\ 
et & Thermal expansivity & $[1/K]$ \\ 
kt & Thermal conductivity & $[Wt/m/K]$ \\
ht  & Heat sources & $[Wt/kg]$ \\
\bottomrule
\end{tabular}
\caption{prn-file Node information block}
\label{tbl:prn_node_info}
\end{table}

\subsection{Part IV: Gridline positions}

\paragraph{Position of gridlines in x direction} $xnumx$ numbers are given
\paragraph{Position of gridlines in y direction }$ynumy$ numbers are given
\paragraph{Position of gridlines in z direction} $znumz$ numbers are given

\subsection{Part V: Boundary Condition Equations}

For each boundary condition there are $5$ values needed. The total number of boundary conditions is $bondnum$. The boundary conditions are all of the following general form:
$$CURPAR = CONST + KOEF1*PAR1$$

\begin{table}[H]
\small
\centering
\begin{tabular}{l p{9cm}}
\toprule
Parameter & Description\\
\midrule
m2 & Index in Indexmatrix bondm\\ 
m3 & Index in BC-matrices, saved in bondm \\ 
bondv1[m3][0] &  $CONST$ value in BC equ. \\ 
bondv1[m3][1] &  $KOEF1$ value in BC equ. \\ 
bondn1[m3] & $PAR1+1$ value in BC equ., $PAR1=0$ means no boundary \\ 
\bottomrule
\end{tabular}
\caption{prn-file boundary condition information block}
\label{tbl:prn_bond_info}
\end{table}

\subsection{Part VI: Markers}

The total number of markers is $marknum$. For each marker the following parameters is given:

\begin{table}[H]
\small
\centering
\begin{tabular}{l p{8cm}}
\toprule
Parameter & Description\\
\midrule
markx & Marker position in x \\ 
marky & Marker position in y \\ 
markz & Marker position in z \\ 
markk & Temperature in K \\ 
markw & Water percentage \\ 
markd & Density \\ 
markex & Accumulated meltfraction \\ 
marktm & Creation time \\ 
markc1 & Melt composition: Granitic part \\ 
markc2 & Melt composition: Dacite part \\
markt  & Rock type \\
\bottomrule
\end{tabular}
\caption{prn-file Marker information block}
\label{tbl:prn_marker_info}
\end{table}